%==============================================================================%
%                PRØVE | FORKURS 1P-2P LÆRERUTDANNING | H2018                  %
%==============================================================================%
%
% __/\\\\\\\\\\\\____________________/\\\\\\______________/\\\\\\\\\_____
%  _\/\\\////////\\\_________________\////\\\____________/\\\///////\\\___
%   _\/\\\______\//\\\___________________\/\\\___________\///______\//\\\__
%    _\/\\\_______\/\\\_____/\\\\\\\\_____\/\\\_____________________/\\\/___
%     _\/\\\_______\/\\\___/\\\/////\\\____\/\\\__________________/\\\//_____
%      _\/\\\_______\/\\\__/\\\\\\\\\\\_____\/\\\_______________/\\\//________
%       _\/\\\_______/\\\__\//\\///////______\/\\\_____________/\\\/___________
%        _\/\\\\\\\\\\\\/____\//\\\\\\\\\\__/\\\\\\\\\_________/\\\\\\\\\\\\\\\_
%         _\////////////_______\//////////__\/////////_________\///////////////_
%
%==============================================================================%
%                               MED HJELPEMIDLER                               %
%==============================================================================%
\Del*{m}


%==============================================================================%
%                                 OPPGAVE 2.1                                  %
%==============================================================================%
\Oppgave[6]

\begin{figure}[H]
  \centering
  \tikzsetnextfilename{Forkurs-1p-2p-laererutdanning-2018-H-oppgave-2.1}
  \begin{tikzpicture}[scale=0.5]
    \draw[black, dashed, ultra thick, color=maincolorDark]   plot[smooth,domain=0:20] (\x,
    {-0.1*\x*\x+2*\x});
    \begin{scope}[scale=2,shift={(-0.6,0.75)}]
      \fill[gray!30!white] (1.2,0.33) circle (0.32);
      \clip (1.2,0.33) circle (0.32);
      \fill[black] (1.06,0.30) -- (1.01,0.17) -- (1.14,0.08) -- (1.26,0.14) -- (1.20,0.28) -- cycle (1.37,0.14) -- (1.46,0.27) -- (1.59,0.27) -- (1.41,0.04) -- cycle (1.28,0.38) -- (1.22,0.52) -- (1.33,0.61) -- (1.45,0.51) -- (1.43,0.37) -- cycle (0.87,0.44) -- (1.02,0.40) -- (1.10,0.53) -- (1.07,0.62) -- (0.94,0.57) -- cycle;
    \end{scope}
    \fill[green!60!black] plot[smooth,domain=-0.25:20.25,samples=501] (\x,0.2+0.3*rnd) |- (-0.25,-0.25);
    \path[rounded corners=1mm,ultra thick, laser beam=maincolorLight,line width=2pt] (-0.25,-0.25) -- (20.25,-0.25) -- (20.25,12.25) --
    (-0.25,12.25) -- cycle;
  \end{tikzpicture}
  \caption{}
  \label{fig:del-2-oppgave-2-1}
\end{figure}

Arild sparket en ball. Ballen fulgte tilnærmet grafen til funksjonen $f$ gitt ved
%
\begin{equation*}
  f(x) = \num{-0.1}x^2 + 2x \qquad , \qquad 0 \leq x \leq 20.
\end{equation*}
%
Når ballen hadde tilbakelagt en horisontal avstand på $x$ meter, var den $f(x)$
meter over bakken.

\begin{oppgaver}
  \Item{2} Bruk graftegner til å tegne grafen til $f$.
\end{oppgaver}

\begin{oppgaver}
  \Item{2} Bestem toppunktet på grafen til $f$. \\
    Hvilken praktisk informasjon gir koordinatene til dette punktet?
\end{oppgaver}

\begin{oppgaver}
  \Item{2} Hvor langt beveget ballen seg i horisontal retning mens den var mer
    enn fem meter over bakken?
\end{oppgaver}

%==============================================================================%
%                                 OPPGAVE 2.2                                  %
%==============================================================================%
\Oppgave[4] \points*{4}

\begin{table}[H]
  \centering
  \caption{}
  \label{tab:del-2-oppgave-2.2}
  \begin{tabular}{|l | S[table-format=2.0]|}
    \tableHeaders{Lengde ($\si{\km}$)}{Antall uker}
    $\left[\phantom{00}0,120\right\rangle$ & 60 \\
    $\left[120,180\right\rangle$ & 60 \\
    $\left[180,240\right\rangle$ & 60 \\
    $\left[240,360\right\rangle$ & 20 \\
    $\left[360,540\right\rangle$ & 20 \\
    \hline
  \end{tabular}
\end{table}

Et idrettslag har $200$ aktive medlemmer. \Cref{tab:del-2-oppgave-2.2} ovenfor
viser hvor mye tid medlemmene brukte på trening i løpet av en uke. \bigskip

Bestem gjennomsnitt og median for det klassedelte datamaterialet.

\clearpage
%==============================================================================%
%                                 OPPGAVE 2.3                                  %
%==============================================================================%
\Oppgave[6]

\begin{figure}[H]
  \centering
  \tikzsetnextfilename{Forkurs-1p-2p-laererutdanning-2018-H-oppgave-2.3}
  \begin{tikzpicture}[scale=0.5]
    \def\ASB{60} \def\CSD{\ASB} \def\radius{8}
    \pgfmathsetmacro{\angleC}{(90-\ASB)*0.5}

    \tkzDefPoint(0,0){S} % Center of circle

    \tkzDefPoint(-\radius,0){s1}
    \tkzDefPoint(\radius,0){s2}
    \tkzDefPoint(0,\radius){s3}
    \tkzDefPoint(0,-\radius){s4}

    \tkzDefPointBy[rotation = center S angle 45](s2) \tkzGetPoint{s5}
    \tkzDefPointBy[rotation = center S angle 180 ](s5) \tkzGetPoint{s6}

    \tkzDefPointBy[rotation = center S angle \angleC](s2) \tkzGetPoint{C}
    \tkzDefPointBy[rotation = center S angle \ASB](C) \tkzGetPoint{D}

    \tkzDrawArc[color=black,dashed](S,s2)(s3)
    \tkzInterLL(S,s5)(C,D) \tkzGetPoint{h2}
    \tkzDrawPolygon[fill=maincolorLight](S,C,D)
    \tkzMarkRightAngle[scale=4](C,h2,S)
    \tkzMarkAngle[scale=1.25](C,S,D)
    \tkzDrawSegment[dashed](S,h2)

    \tkzDefPointBy[rotation = center S angle 180](C) \tkzGetPoint{A}
    \tkzDefPointBy[rotation = center S angle 180](D) \tkzGetPoint{B}

    % Draw the triangles with \DSA angle
    \tkzDrawArc[color=black,dashed](S,s1)(s4)
    \tkzInterLL(S,s6)(A,B) \tkzGetPoint{h1}
    \tkzDrawPolygon[fill=maincolorLight](S,A,B)
    \tkzMarkRightAngle[scale=4](S,h1,B)
    \tkzMarkAngle[scale=1.25](A,S,B)
    \tkzDrawSegment[dashed](S,h1)

    % Draw the 90 degree circle sectors
    \tkzDrawSector[rotate,fill=maincolorLight](S,s1)(-90)
    \tkzDrawSector[rotate,fill=maincolorLight](S,s2)(-90)
    \tkzMarkRightAngle[scale=5](s3,S,s1)
    \tkzMarkRightAngle[scale=5](s4,S,s2)

    % Label points
    \tkzLabelPoint[left](A){$A$}
    \tkzLabelPoint[below](B){$B$}
    \tkzLabelPoint[right](C){$C$}
    \tkzLabelPoint[above](D){$D$}
    \tkzLabelPoint[below right](S){$S$}

    \tkzLabelSegment[below right](S,h1){$h_1$}
    \tkzLabelSegment[below right](S,h2){$h_2$}
  \end{tikzpicture}
  \caption{}
  \label{fig:del-2-oppgave-2.4}
\end{figure}

Sirkelen i figuren ovenfor har sentrum i $S$ og radius $\SI{8.0}{\cm}$. \\
$\angle ASB = \angle CSD = \SI{60}{\degree}$.

\begin{oppgaver}
  \Item{2} Bestem samlet omkrets av de fargelagte områdene i figuren.
\end{oppgaver}

\begin{oppgaver}
  \Item{2} Vis at høydene $h_1$ og $h_2$ har lengde $\SI{6.9}{\cm}$.
\end{oppgaver}

\begin{oppgaver}
  \Item{2} Bestem samlet areal av de fargelagte områdene i figuren.
\end{oppgaver}


%==============================================================================%
%                                 OPPGAVE 2.4                                  %
%==============================================================================%
\Oppgave[6]


I en stor kommune skal $\num{1000}$ innbyggere testes for å finne ut om de har
en bestemt sykdom. De $\num{1000}$ innbyggerne trekkes ut tilfeldig. Fra
tidligere undersøkelser vet man at

\begin{itemize}
  \item $\SI{ 1}{\percent}$ av alle personer har denne sykdommen.
  \item $\SI{80}{\percent}$ av personene som har denne sykdommen, får positivt
    utslag på testen.
  \item $\SI{10}{\percent}$ av personene som ikke har denne sykdommen, får
    positivt utslag på testen.
\end{itemize}

Vi antar at tallene gjelder for de $\num{1000}$ innbyggerne som er
trukket ut.

\begin{oppgaver}
  \Item{2} Tegn av og fyll ut krysstabellen nedenfor.
\end{oppgaver}

\begin{table}[H]
  \caption{}
  \label{table:del-2-oppgave-2.4}
  \begin{tabular}{| l | c | c | c |}
                   \hline \Rowcolor & {Syk\headerstrut} & {Ikke syk} & {Sum}    \\ \hline
    \Cellcolor{\headerstrut%
    Positivt utslag på testen}      &                   &            &          \\ \hline
    \Cellcolor{\headerstrut%
    Ikke positivt utslag på testen} &                   &            &          \\ \hline
    \Cellcolor{\headerstrut%
    Sum}                            &                   &            &\num{1000}\\ \hline
  \end{tabular}
\end{table}

\begin{oppgaver}
  \Item{1} Bestem sannsynligheten for at en person som er trukket ut, ikke har
  denne sykdommen
\end{oppgaver}

\begin{oppgaver}
  \Item{1} Bestem sannsynligheten for at en person som er trukket ut, får
  positivt utslag på testen.
\end{oppgaver}

Tenk deg at en person som er trukket ut, får positivt utslag på testen.

\begin{oppgaver}
  \Item{2} Bestem sannsynligheten for at denne personen virkelig har sykdommen.
\end{oppgaver}


%==============================================================================%
%                                 OPPGAVE 2.5                                  %
%==============================================================================%
\Oppgave[8]


\begin{figure}[H]
  \centering
  \tikzsetnextfilename{Forkurs-1p-2p-laererutdanning-2018-H-oppgave-2.5}
  \begin{tikzpicture}[scale=0.25]
    \tkzInit[xmin=-7,xmax=40.5,ymin=-0.5,ymax=29.5]
    \tkzClip

    \def\height{24}
    \def\width{40}
    \def\sides{3}
    \def\textSpace{5}

    \tkzDefPoint(-\textSpace,0){u}\tkzDefPoint(-\textSpace,\height){v}
    \tkzDefPoint(0,\textSpace+\height){w}\tkzDefPoint(\width,\textSpace+\height){z}

    \tkzDefPoint(0,0){A}\tkzDefPoint(\width,0){B}
    \tkzDefPoint(\width,\height){C}\tkzDefPoint(0,\height){D}

    \tkzDefPoint(\sides,\sides){a}\tkzDefPoint(\width-\sides,\sides){b}
    \tkzDefPoint(\width-\sides,\height-\sides){c}\tkzDefPoint(\sides,\height-\sides){d}

    \tkzDrawSegments[|-|](u,v w,z)

    \draw[thick,color=maincolorMedium] (A) rectangle (a);
    \draw[thick,color=maincolorMedium] (B) rectangle (b);
    \draw[thick,color=maincolorMedium] (C) rectangle (c);
    \draw[thick,color=maincolorMedium] (D) rectangle (d);

    \tkzDrawPolygon[dashed,thick,maincolorMedium](a,b,c,d)
    \tkzDrawPolygon[thick,maincolorMedium](A,B,C,D)

    \midlabelline{w}{z}{$\SI{\width}{\cm}$}
    \midlabelline{u}{v}{$\SI{\height}{\cm}$}
  \end{tikzpicture}
  \caption{}
  \label{fig:del-2-oppgave-2.5}
\end{figure}

Tenk deg at du skal lage en eske av en papplate. Papplaten har form som et
rektangel med lengde $\SI{40}{\cm}$ og bredde $\SI{24}{\cm}$. For å lage esken
skal du klippe bort et kvadrat i hvert hjørne av papplaten og brette langs de
stiplede linjene. Se \cref{fig:del-2-oppgave-2.5} ovenfor.

\begin{oppgaver}
  \Item{1} Bestem volumet av esken dersom sidene i kvadratene du klipper bort,
  er $\SI{3}{\cm}$.
\end{oppgaver}

Sett lengden av sidene i kvadratene du klipper bort, lik $x~\si{\cm}$.

\begin{oppgaver}
  \Item{2} Vis at volumet $V(x)~\si{\cm\cubed}$ av esken da kan skrives som
    %
    \begin{equation*}
      V(x) = 4x^3 - 128x^2 + 960x
    \end{equation*}
    %
    og forklar at $0 < x < 12$.
\end{oppgaver}

\begin{oppgaver}
  \Item{2} Bruk graftegner til å tegne grafen til $V$.
\end{oppgaver}

\begin{oppgaver}
  \Item{2} Hvor store må sidene i kvadratene du klipper bort, være for at esken
    skal få størst mulig volum? Hvor stort volum får esken da?
\end{oppgaver}


%==============================================================================%
%                                 OPPGAVE 2.6                                  %
%==============================================================================%
\Oppgave[6]

Miriam har fått en infeksjon og skal ta tabletter med et virkestoff mot
infeksjonen. Én tablett inneholder $\SI{120}{\milli\gram}$ virkestoff. Vi antar
at antall milligram virkestoff i kroppen reduseres med $\SI{3}{\percent}$ hver
time.

\begin{oppgaver}
  \Item{1} Hvor mange milligram av virkestoffet vil være igjen i kroppen én time
    etter at Miriam har tatt den første tabletten?
\end{oppgaver}

\begin{oppgaver}
  \Item{1} Hvor mange milligram av virkestoffet vil være igjen i kroppen
    $10$~timer etter at Miriam har tatt den første tabletten?
\end{oppgaver}

Miriam skal ta én tablett hver $12$. time i $14$ døgn.

\begin{oppgaver}
  \Item{2} Lag et regneark som viser hvor mange milligram av virkestoffet hun
    vil ha i kroppen rett før og rett etter at hun tar en ny tablett disse $14$
    døgnene.
\end{oppgaver}

\begin{oppgaver}
  \Item{2} Hva er laveste og høyeste antall milligram Miriam vil ha i kroppen i
    perioden fra hun har tatt den første tabletten, til rett etter at hun har
    tatt den siste?
\end{oppgaver}
