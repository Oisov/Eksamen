%==============================================================================%
%            PRØVE | FORKURS 1P-2P LÆRERUTDANNING | V2017 | UTSATT             %
%==============================================================================%
%
% __/\\\\\\\\\\\\____________________/\\\\\\___________________/\\\_
%  _\/\\\////////\\\_________________\////\\\_______________/\\\\\\\_
%   _\/\\\______\//\\\___________________\/\\\______________\/////\\\_
%    _\/\\\_______\/\\\_____/\\\\\\\\_____\/\\\__________________\/\\\_
%     _\/\\\_______\/\\\___/\\\/////\\\____\/\\\__________________\/\\\_
%      _\/\\\_______\/\\\__/\\\\\\\\\\\_____\/\\\__________________\/\\\_
%       _\/\\\_______/\\\__\//\\///////______\/\\\__________________\/\\\_
%        _\/\\\\\\\\\\\\/____\//\\\\\\\\\\__/\\\\\\\\\_______________\/\\\_
%         _\////////////_______\//////////__\/////////________________\///_
%
%==============================================================================%
%                              UTEN HJELPEMIDDEL                               %
%==============================================================================%
\Del{u}


%==============================================================================%
%                                 OPPGAVE 1.1                                  %
%==============================================================================%
\Oppgave[1] \points*{1}

Regn ut og skriv svaret på standardform
%
\begin{equation*}
  \num{8.4e7} - \num{5.0e6}
\end{equation*}


%==============================================================================%
%                                 OPPGAVE 1.2                                  %
%==============================================================================%
\Oppgave[5]

Noen elever undersøkte hvor mange personer det var i hver enkelt bil som
parkerte ved skolen en bestemt dag. Resultatet av undersøkelsen er vist i
\cref{tab:Forkurs-2017-H-oppgave-1.2} nedenfor.

\begin{table}[H]
    \centering
    \caption{}
    \label{tab:Forkurs-1p-2p-laererutdanning-2017-V-U-oppgave-1-2}
    \begin{tabular}{|S[table-format=1.0]|S[table-format=2.0]|}
         \tableHeaders{Antall personer i bilen}{Antall biler}
         1 &  83 \\
         2 &  22 \\
         3 &   9 \\
         4 &   4 \\
         5 &   2 \\ \hline
    \end{tabular}
\end{table}

%==============================================================================%
%                                 OPPGAVE 1.3                                  %
%==============================================================================%
\Oppgave[3]

$2015$ er nå satt som basisår for konsumprisindeksen. I $130$ var
konsumprisindeksen $\num{3,0}$. Lars hadde da en nominell lønn på $3000$ kroner.

\begin{oppgaver}
  \Item{1} Bestem reallønnen til Lars i $1930$.
\end{oppgaver}

\begin{oppgaver}
  \Item{2} Hva måtte din nominelle lønn vært i $2015$ dersom kjøpekraften din
  dete året skulle være lik kjøpekraften til Lars i $1930$?
\end{oppgaver}

%==============================================================================%
%                                 OPPGAVE 1.4                                  %
%==============================================================================%
\Oppgave[3]

Noen venner vil dra på tur i høstferien. De vil leie en hytte og en båt. Det
koster $2000$ kroner å leie hytten og $1000$ kroner å leie båten. Utgiftene skal
deles likt mellom dem som blir med på turen.

\begin{oppgaver}
  \Item{1} Hvor mange må minst bli med for at hver av dem ikke skal måtte betale
  mer enn 400 kroner?
\end{oppgaver}

\begin{oppgaver}
  \Item{1} Bestem en modell $U(x)$ som viser hvor mye hver person må betale
  dersom $x$ personer blir med.
\end{oppgaver}

\begin{oppgaver}
  \Item{1} Hvilken av de to grafene fra \cref{fig:Forkurs-2017-H-oppgave-2.2}
  beskriver situasjonen ovenfor best? Begrunn svaret ditt.
\end{oppgaver}

\begin{figure}[H]
  \tikzsetnextfilename{Forkurs-1p-2p-laererutdanning-2017-V-U-oppgave-1-4}
  \begin{tikzpicture}
    \def\prisBil{1000}\def\prisBaat{2000}\def\prisTotalt{\prisBil+\prisBaat}
    \def\prisPerPers{400}
    \pgfmathsetmacro{\antallPersoner}{\prisTotalt/\prisPerPers}

    \def\scaleNum{1000}
    \pgfmathsetmacro{\prisBilC}{\prisBil/\scaleNum}
    \pgfmathsetmacro{\prisBaatC}{\prisBaat/\scaleNum}
    \pgfmathsetmacro{\prisTotaltC}{\prisBilC+\prisBaatC}
    \pgfmathsetmacro{\prisPerPersC}{\prisPerPers/\scaleNum}
    \pgfmathsetmacro{\antallPersonerC}{\prisTotaltC/\prisPerPersC}
    \begin{axis}[
      Eksamen1,
      ytick={0,0.5,...,\prisTotaltC+1},
      yticklabel = {\num{\fpeval{\tick*\scaleNum}}},
      xtick={0,...,11},
      xticklabel = {\phantom{\tick}},
      ymin=-.1,
      ymax=\prisTotaltC+1.5,
      xmin=-.1,
      xmax=10.9,
      domain = 1:\antallPersonerC,
      ]
      \addplot[color=maincolorMedium!30!white,thick,samples=50] {\prisTotaltC/x}
        node[below=1cm,pos=0.2] {Graf $2$};
      \addplot[color=maincolorMedium,thick,samples=50] {\prisTotaltC + (1 -
        x)*\prisPerPersC} node[above right,pos=0.5] {Graf $1$};
      \node[] at (axis cs: 3,4) {Pris per person (kroner)};
      \node[] at (axis cs: 9,0.75) {Antall personer};
    \end{axis}
  \end{tikzpicture}
  \caption{}
  \label{fig:Forkurs-1p-2p-laererutdanning-2017-V-U-oppgave-1-4}
\end{figure}


%==============================================================================%
%                                 OPPGAVE 1.5                                  %
%==============================================================================%
\Oppgave[5]

En bokklubb har i dag $\num{8000}$. Det er satt opp to ulike modeller for å
beskrive hvordan medlemstallet vil endre seg de neste årene.

Ifølge modell $A$ vil det om $x$ år være $A(x)$ medlemmer i bokklubben, der
%
\begin{equation*}
  A(x) = \num{1200}x + \num{8000}
\end{equation*}
%
Ifølge modell $B$ vile det om $x$ være $B(x)$ medlemmer i bokklubben, der
%
\begin{equation*}
  B(x) = \num{8000} \cdot \num{1.15}^x
\end{equation*}
%
\begin{oppgaver}
  \Item{2} Beskriv hvordan medlemstallet vil endre seg ifølge modell $A$ og ifølge
  modell $B$.
\end{oppgaver}

\begin{oppgaver}
  \Item{1} Bestem $A(0)$, $B(0)$, $A(1)$ og$B(1)$.
\end{oppgaver}

\begin{oppgaver}
  \Item{2} Skisser grafen til $A$ og grafen til $B$ i samme koordinatsystem.
\end{oppgaver}


%==============================================================================%
%                                 OPPGAVE 1.6                                  %
%==============================================================================%
\Oppgave[2] \points*{2}


Ved en skole er $\SI{40}{\percent}$ av elevene gutter. $\SI{50}{\percent}$ av
jentene spiller håndball. Ingen gutter spiller håndball. \bigskip

Hvor mange prosent av elevene ved skolen spiller håndball?


%==============================================================================%
%                                 OPPGAVE 1.7                                  %
%==============================================================================%
\Oppgave[5]

Ovenfor ser du tre figurer. Figurene er satt sammen av små kvadrater. Hvert
kvadrat har areal $1$. Tenk deg at du skal fortsette å lage figurer etter samme
mønster.

\begin{oppgaver}
  \Item{1} Bestem omkretsen av $(4)$.
\end{oppgaver}

\begin{oppgaver}
  \Item{1} Bestemet uttrykk for omkretsen av $(n)$, uttrykt ved $n$.
\end{oppgaver}

En figur som følger samme mønster som ovenfor har en omkrets på $1024$.

\begin{oppgaver}
  \Item{2} Hvor mange rader med små kvadrater er det i denne figuren?
\end{oppgaver}

