
%% =========================================
%%   DEL 2 - MED HJELPEMIDLER
%% =========================================

\Del{m}

\Oppgave[4] % Oppgave 2.1

\begin{table}[H]
    \centering
    \caption{}
    \begin{tabularx}{\textwidth}{| X | *{2}{S[table-format=2.0]|}}
        \tableHeader{3}{Stortinget ved starten av perioden $2017$--$2021$}
    \Rowcolor
        Parti  \headerstrut & {Antall kvinner} & {Antall menn} \\
    \hline
        Arbeiderpartiet           &        24        &      25 \\
    \hline
        Høyre                     &        20        &      25 \\
    \hline
        Fremskrittspartiet        &         7        &      20 \\
    \hline
        Senterpartiet             &        10        &       9 \\
    \hline
        Sosialistisk Venstreparti &         4        &       7 \\
    \hline
        Kristelig Folkeparti      &         2        &       6 \\
    \hline
        Venstre                   &         1        &       7 \\
    \hline
        Miljøpartiet De Grønne    &         1        &         \\
    \hline
        Rødt                      &                  &       1 \\
    \hline
    \end{tabularx}
    \label{tab:del-1-oppgave-2.1}
\end{table}

\Cref{tab:del-1-oppgave-2.1} viser stortingsrepresentantene fordelt på parti og
kjønn etter stortingsvalget $2017$.

\begin{oppgaver}
    \Item{2} Legg tabellen inn i et regneark, og bruk regnearket til å lage et
    diagram som illustrerer opplysningene som er gitt.
\end{oppgaver}

\begin{oppgaver}
    \Item{2} Lag en ny kolonne i regnearket som viser prosentandelen kvnner i
    hvert parti.
\end{oppgaver}


\Oppgave[6] % Oppgave 2.2

viser indeksen for en vare noen år i perioden $2000$-$2017$

La $x=0$ svare til år $2000$, $x=5$ til år $2005$, og så videre

\begin{oppgaver}
    \Item{2} bruk regresjon til å vise at funksjonen $f$ er gitt ved
    %
    \begin{equation*}
        f(x) = \num{0.01} x^3 - \num{0.52}x^2 + \num{7.15}x + 75
    \end{equation*}
    %
    er en modell som passer godt med tallene i tabellen.
\end{oppgaver}

\begin{oppgaver}
    \Item{2} Bestem den gjennomsnittlige vekstfarten til funksjonen $f$ fra $x =
    1$ til $x = 4$.  Gi en praktisk tolkning av dette svaret.
\end{oppgaver}

\begin{oppgaver}
    \Item{2} Bestem den momentane vekstfarten til funksjonen $f$ når $x = 12$.
    Gi en praktisk tolkning av dette svaret.
\end{oppgaver}


\Oppgave[4] % Oppgave 2.3

I en konfekteske er det $25$ sjokoladebiter, Jan liker $15$ av disse bitene.
Pernille tar tilfeldig to biter fra esken og gir dem til Jan.

\begin{oppgaver}
    \Item{2} Bestem sannsynligheten for at Jan liker begge bitene.
\end{oppgaver}

\begin{oppgaver}
    \Item{2} Bestem sannsynligheten for at Jan liker nøyaktig èn av bitene.
\end{oppgaver}


\Oppgave[4] % Oppgave 2.4

Anders og Lotte bruker Snapchat. Nedenfor ser du hvor mange \enquote{streaks}
Anders ar med ti av vennene sine.

\begin{oppgaver}
    \Item{2} Bestem gjennomsnittet og standardavviket for antall
    \enquote{streaks} Anders har med disse ti vennene.
\end{oppgaver}

Lotte har beregnet gjennomsnittet og standardavviket for antall
\enquote{streaks} hun har med ti av sine venner. Hun fikk et lavere gjenomsnitt
enn Anders, men et høyere standardavvik.

\begin{oppgaver}
    \Item{2} Nedenfor er det satt opp tre påstander. Avgjør om hver enkelt
    påstand \textbf{kan} være riktig. Begrunn svarene dine.
\end{oppgaver}


\Oppgave[4] % Oppgave 2.5

Ovenfor ser du en figur tegnet på et rutenett. Anta at hver rute er kvadratisk
med side $\SI{1}{\cm}$.

\begin{oppgaver}
    \Item{2} Bestem arealet av det fargelagte området.
\end{oppgaver}

\begin{oppgaver}
    \Item{2} Bestem omkretsen av det fargelagte området.
\end{oppgaver}


\Oppgave[6] % Oppgave 2.6

For nøyaktig fem år siden satte Kari inn $\num{25000}$ kroner på en sparekontor.
Pengene har stått urørt. Kontoen har en fast årlig rente på
$\SI{2.5}{\percent}$.

\begin{oppgaver}
    \Item{2} Hvor mye har Kari på sparekonto i dag?
\end{oppgaver}

Kari vurderer å la pengene fortsatt stå urørt på kontoen.

\begin{oppgaver}
    \Item{2} Hvor mange år vil det da gå fra hun satte inn pengene, til hun har
    $\num{50000}$~kroner på konto om fire år?
\end{oppgaver}

Kari bestemmer seg for å sette inn mer penger på kontoen.

\begin{oppgaver}
    \Item{2} Hvor mye må hun sette inn på sparekontoen i dag for at det skal stå
    $\num{50000}$~kroner på kontoen om $4$~år?
\end{oppgaver}


\Oppgave[3] % Oppgave 2.7

Jotun Husvask skal blandes med vann i forholdet $1:20$.

\begin{oppgaver}
    \Item{1} Lars har en bøtte med $\SI{5}{\L}$ vann. Hvor mange desiliter må
    han tilsette?
\end{oppgaver}

Lise har $\SI{6.3}{\L}$ ferdig blanding i forholdet $1:20$, men ønsker å
tilsette mer Husvask slit at blandingsforholdet blir $1:15$.

\begin{oppgaver}
    \Item{2} Hvor mange desiliter Husvask må hun tilsette?
\end{oppgaver}

\Oppgave[5] % Oppgave 2.8
\points*{5}

\begin{table}[htbp]
    \centering
    \caption{}
    \begin{tabular}{|l S[table-format=4.0] S[table-format=6.0,] l S[table-format=6.0] l|}
    \hline \Rowcolor
         Navn   &
         {Fødselsår} &
         \multicolumn{2}{c}{\vtop{\hbox{\strut Årslønn i $2017$}\hbox{\strut
   inkludert feriepenger}}} & \multicolumn{2}{c|}{Feriepenger i $2017$} \\
    \hline
         Mari   & 1970    &  734567 & kroner  &  76661 & kroner \\
         Morten & 1998    &  430124 & kroner  &  45972 & kroner \\
         Stein  & 1982    &  649345 & kroner  &  66540 & kroner \\
         Inger  & 1957    &  385433 & kroner  &  40902 & kroner \\
    \hline
    \end{tabular}
    \label{tab:my_label}
\end{table}

Ovenfor ser du de første linjene i en tabell fra regnskapsavdelingen i en
bedrift. \bigskip

Lag et regneark som vist nedenfor. Registrer opplysningene fra tabellen i de
hvite cellene i regnearket, og sett inn formler i de fargede cellene. \bigskip

Feriepengesatsen er $\SI{12.0}{\percent}$ for arbeidstakere under $60$ år og
$\SI{14.3}{\percent}$ for arbeidstakere over $60$ år.



