%==============================================================================%
%            PRØVE | FORKURS 1P-2P LÆRERUTDANNING | V2017 | UTSATT             %
%==============================================================================%
%
% __/\\\\\\\\\\\\____________________/\\\\\\______________/\\\\\\\\\_____
%  _\/\\\////////\\\_________________\////\\\____________/\\\///////\\\___
%   _\/\\\______\//\\\___________________\/\\\___________\///______\//\\\__
%    _\/\\\_______\/\\\_____/\\\\\\\\_____\/\\\_____________________/\\\/___
%     _\/\\\_______\/\\\___/\\\/////\\\____\/\\\__________________/\\\//_____
%      _\/\\\_______\/\\\__/\\\\\\\\\\\_____\/\\\_______________/\\\//________
%       _\/\\\_______/\\\__\//\\///////______\/\\\_____________/\\\/___________
%        _\/\\\\\\\\\\\\/____\//\\\\\\\\\\__/\\\\\\\\\_________/\\\\\\\\\\\\\\\_
%         _\////////////_______\//////////__\/////////_________\///////////////_
%
%==============================================================================%
%                               MED HJELPEMIDLER                               %
%==============================================================================%
\Del*{m}


%==============================================================================%
%                                 OPPGAVE 2.1                                  %
%==============================================================================%
\Oppgave[3]

Det er omtrent $\num{5.3}$~millioner innbyggere i Norge. I gjennomsnitt kaster
hver innbygger $180$~plastposer hvert år. Normal tykkelse på plasten i en pose
er $\SI{0.035}{\milli\metre}$. \bigskip

Tenk deg at vi legger alle disse plastposene oppå hverandre i en stabel.

\begin{oppgaver}
  \Item{1} Omtrent hvor høy ville stabelen bli?
\end{oppgaver}

Norges høyeste fjell, Galdhøpiggen, er $\SI{2469}{\m}$.

\begin{oppgaver}
  \Item{2} Hvor mange dager ville det gå før stabelen var like høy som
  Galdhøpiggen, dersom vi antar det kastes like mange poser hver dag?
\end{oppgaver}


%==============================================================================%
%                                 OPPGAVE 2.2                                  %
%==============================================================================%
\Oppgave[4]

Figuren viser en $20$-sidet terning der tallene $1$, $2$, $3$, $20$ er skrevet
på sidene. Når du kaster terningen, er hvert av de $20$ utfallene like
sannsynlig.  Tenk deg at du skal kaste terningen to ganger.

\begin{oppgaver}
  \Item{1} Bestem sannsynligheten for at du kommer til å få tallet $20$ i første
  kast.
\end{oppgaver}

\begin{oppgaver}
  \Item{1} Bestem sannsynligheten for at du kommer til å få tallet $20$ i begge
  kastene.
\end{oppgaver}

\begin{oppgaver}
  \Item{2} Bestem sannsynligheten for at summen av tallene du får i første og
  andre kast, blir mindre enn seks.
\end{oppgaver}



%==============================================================================%
%                                 OPPGAVE 2.3                                  %
%==============================================================================%
\Oppgave[4]

\begin{figure}[H]
  \centering
  \begin{tikzpicture}
    \def\radius{2}
    \pgfmathsetmacro{\ED}{2*\radius}
    \pgfmathsetmacro{\MC}{2*\radius}
    \def\AM{\ED} \def\MB{\AM}

    \tkzDefPoint(0,0){A}
    \tkzDefPoint(\AM,0){M}
    \tkzDefPoint(\AM+\MB,0){B}
    \tkzDefPoint(\AM,\radius){S}
    \tkzDefPoint(\AM,\MC){C}

    \tkzDefMidPoint(A,C) \tkzGetPoint{E}
    \tkzDefMidPoint(B,C) \tkzGetPoint{D}

    \tkzDrawPolygon[color=maincolorDark, ultra thick, fill=maincolorLight](A,B,C)
    \tkzDrawSector[color=maincolorDark, thick,fill=maincolorMedium](S,E)(D)

    \tkzLabelPoint[below left](A){$A$}
    \tkzLabelPoint[below right](B){$B$}
    \tkzLabelPoints[above](S,C)
    \tkzLabelPoint[below](M){$M$}
    \tkzLabelPoint[above left](E){$E$}
    \tkzLabelPoint[above right](D){$D$}
  \end{tikzpicture}
  \caption{}
  \label{fig:Forkurs-1p-2p-laererutdanning-2017-V-U-oppgave-2-3}
\end{figure}

\Cref{fig:Forkurs-1p-2p-laererutdanning-2017-V-U-oppgave-2-3} ovenfor viser en
likebeint trekant $ABC$.$M$ er midtpunkt på $AB$. I trekanten er det innskrevet
en halvsirkel med sentrum i $S$ og radius $2$. Punktet $D$ ligger på $BC$, og
punktet $E$ ligger på $AC$. $AB\parallel ED$ og $AM=ED$.

\begin{oppgaver}
    \Item{2} Forklar at $\Delta ABC$ og $\Delta EDC$ er formlike.
\end{oppgaver}

\begin{oppgaver}
    \Item{1} Forklar at $CM = 2 CS$.
\end{oppgaver}

\begin{oppgaver}
    \Item{2} Gjør beregninger og avgjør om arealet av halvsirkelen er mindre enn
      halvparten av det samlede arealet av de lys lilla områdene.
\end{oppgaver}


%==============================================================================%
%                                 OPPGAVE 2.4                                  %
%==============================================================================%
\Oppgave[5]

\begin{figure}[H]
  \centering
  \ballgolf{5}
  \caption{}
  \label{fig:Forkurs-1p-2p-laererutdanning-2017-V-U-oppgave-2-4}
\end{figure}

I \cref{fig:Forkurs-1p-2p-laererutdanning-2017-V-U-oppgave-2-4} ovenfor ser du
15 kuler. Hver kule har diameter $\SI{1.0}{\cm}$.

\begin{oppgaver}
  \Item{} Bestem det totale volumet av alle kulene, $V_K$.
\end{oppgaver}

Til høyre ovenfor ser du $15$ sirkler. Hver sirkel har diameter $\SI{1.0}{\cm}$.
La $V_T$ være volumet av et rett trekantet prisme med grunnflate lik det samlede
arealet av sirklene og høyde $\SI{1.0}{\cm}$.

\begin{oppgaver}
  \Item{} Bestem forholdet mellom $V_T$ og $V_K$.
    \label{delopg:Forkurs-1p-2p-laererutdanning-2017-V-U-oppgave-2-4}
\end{oppgaver}

\begin{oppgaver}
  \Item{} Hvor høyt må det rette trekantede prismet være for at forholdet i
    \cref{delopg:Forkurs-1p-2p-laererutdanning-2017-V-U-oppgave-2-4} skal bli
    lik $1$?
\end{oppgaver}

%==============================================================================%
%                                 OPPGAVE 2.5                                  %
%==============================================================================%
\Oppgave[9]

\Cref{tab:Forkurs-1p-2p-laererutdanning-2017-V-U-oppgave-2-5} nedenfor viser
hvor mange personer i Norge som var $100$~år eller eldre $1$.januar noen
utvalgte år.

\begin{table}[H]
    \caption{}
    \label{tab:Forkurs-1p-2p-laererutdanning-2017-V-U-oppgave-2-5}
    \begin{tabularx}{\textwidth}{|c| *{7}{Y|}} \hline \Rowcolor
        år               &     1     &     2     &     4     &     6     &     8     &     10    &     12    \\ \hline
        Antall personer & & & & & & & \\
        som var $100$ år & \num{432} & \num{467} & \num{544} & \num{641} &
        \num{736} & \num{886} & \num{942} \\
        eller eldre  & & & & & & & \\ \hline
    \end{tabularx}
\end{table}

\begin{oppgaver}
  \Item{2} Bruk regresjon til å vise at funksjonen $f$ gitt ved
  %
  \begin{equation*}
    f(x) = 402 \cdot \num{1.054}^x
  \end{equation*}
  %
  er en god modell for antall personer som var $100$~år eller elde $x$~år etter
  $1$.januar år~$2000$.
\end{oppgaver}

\begin{oppgaver}
    \Item{2} Bruk graftegner til å tegne grafen til $f$ for $1 \leq x \leq 20$
\end{oppgaver}

\begin{oppgaver}
    \Item{1} Bestem $f(17)$. Hvilken praktisk informasjon gir dette svaret?
\end{oppgaver}

\begin{oppgaver}
    \Item{2} Bestem den gjennomsnittlige vekstfarten til funksjonen $f$ fra
    $x=1$ til $x=16$. Hvilken praktisk informasjon gir dette svaret?
\end{oppgaver}

\begin{oppgaver}
    \Item{2} Bestem den momentane vekstfarten til funksjonen $f$ når $x=16$.
    Hvilken praktisk informasjon gir dette svaret?
\end{oppgaver}


%==============================================================================%
%                                 OPPGAVE 2.6                                  %
%==============================================================================%
\Oppgave[6]

Ved en skole ble $\num{200}$ elever spurt om hvor lang reisetid de har fra
bosted til skole. Se
\cref{tab:Forkurs-1p-2p-laererutdanning-2017-V-U-oppgave-2-6}.

\begin{table}[H]
  \caption{}
  \label{tab:Forkurs-1p-2p-laererutdanning-2017-V-U-oppgave-2-6}
  \begin{tabular}{|c | S[table-format=3.0]|}
    \tableHeaders{Reisetid i minutt}{Frekvens}
    $\left[\phantom{1}0, 10 \right\rangle$ & 20 \\
    $\left[10, 20 \right\rangle$ & 60 \\
    $\left[20, 40 \right\rangle$ & 80 \\
    $\left[40, 80 \right\rangle$ & 40 \\
    \tableHeaders{Totalt}{200}
  \end{tabular}
\end{table}

\begin{oppgaver}
  \Item{} Bestem gjennomsnittet for datamaterialet.
\end{oppgaver}

\begin{oppgaver}
  \Item{} Bestem medianen for datamaterialet.
\end{oppgaver}

\begin{oppgaver}
  \Item{} Lag et histogram som viser fordelingen av reisetider.
\end{oppgaver}


%==============================================================================%
%                                 OPPGAVE 2.7                                  %
%==============================================================================%
\Oppgave[5] \points*{5}

Et idrettslag arrangerer \enquote{Søndagstrimmen} hver uke. Hver
deltaker løper først noen kilometer og er så med på ulike aktiviteter i
idrettshallen. Deltakerne får poeng for hver kilometer de løper, og for
aktivitetene de deltar i. \bigskip

Lokale bedrifter sponser arrangementet med et kronebeløp
(\enquote{sponsorkroner}) tilsvarende en prosentdel av poengene deltakerne
oppnår. \bigskip

Du skal lage et regneark som vist nedenfor. I de hvite cellene skal idrettslaget
registrere opplysninger. I de lilla cellene skal du lage formler. \bigskip

\begin{itemize}
  \item Antall poeng per kilometer og hvor stor prosentdel av poengsummene som
    sponses, varierer fra uke til uke og skal registreres i celle B$4$ og B$5$.
    %
  \item Personer under 16 år regnes som barn. Når alderen registreres, skal
    regnearket automatisk plassere deltakeren i klasse \enquote{Barn} eller
    klasse \enquote{Voksen}.
    %
  \item Samlet poengsum for hver deltaker er summen av poengene for kilometerne
    deltakeren har løpt, og poengene deltakeren har oppnådd i aktivitetene.
\end{itemize}







