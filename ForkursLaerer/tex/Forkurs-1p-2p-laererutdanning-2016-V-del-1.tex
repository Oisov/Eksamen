%==============================================================================%
%                PRØVE | FORKURS 1P-2P LÆRERUTDANNING | V2016                  %
%==============================================================================%
%
% __/\\\\\\\\\\\\____________________/\\\\\\___________________/\\\_
%  _\/\\\////////\\\_________________\////\\\_______________/\\\\\\\_
%   _\/\\\______\//\\\___________________\/\\\______________\/////\\\_
%    _\/\\\_______\/\\\_____/\\\\\\\\_____\/\\\__________________\/\\\_
%     _\/\\\_______\/\\\___/\\\/////\\\____\/\\\__________________\/\\\_
%      _\/\\\_______\/\\\__/\\\\\\\\\\\_____\/\\\__________________\/\\\_
%       _\/\\\_______/\\\__\//\\///////______\/\\\__________________\/\\\_
%        _\/\\\\\\\\\\\\/____\//\\\\\\\\\\__/\\\\\\\\\_______________\/\\\_
%         _\////////////_______\//////////__\/////////________________\///_
%
%==============================================================================%
%                              UTEN HJELPEMIDDEL                               %
%==============================================================================%

\Del{u}


%==============================================================================%
%                                 OPPGAVE 1.1                                  %
%==============================================================================%
\Oppgave[3] \points*{3}

Anders målte temperaturen utenfor hytta $10$ dager i februar.

\begin{table}[H]
    \centering
    \begin{tabular}{|c |S[table-format=1.0]|}
         \tableHeaders{Dato}{Temperatur}
         01.02 &  -8\si{\celsius} \\
         02.02 &  -2\si{\celsius} \\
         03.02 &   4\si{\celsius} \\
         04.02 &   8\si{\celsius} \\
         05.02 &   3\si{\celsius} \\
         06.02 & -12\si{\celsius} \\
         07.02 &  -2\si{\celsius} \\
         08.02 &   3\si{\celsius} \\
         09.02 &   6\si{\celsius} \\
         10.02 &  -2\si{\celsius} \\ \hline
    \end{tabular}
    \caption{}
    \label{tab:del-2-oppgave-2}
\end{table}

Bestem gjennomsnittet, medianen, typetallet og variasjosbredden for
temperaturmålingene.


%==============================================================================%
%                                 OPPGAVE 1.2                                  %
%==============================================================================%
\Oppgave[1] \points*{1}

I en klasse er forholdet mellom antall jenter og antall gutter $3:4$.
Det er $12$ gutter i klassen. \bigskip

Hvor mange elever er det i klassen?


%==============================================================================%
%                                 OPPGAVE 1.3                                  %
%==============================================================================%
\Oppgave[1] \points*{1}

Du får $\SI{40}{\percent}$ rabatt på en vare. Denne rabatten utgjør $200$
kroner.\bigskip

Hvor mye koster varen etter at rabatten er trukket fra?


%==============================================================================%
%                                 OPPGAVE 1.4                                  %
%==============================================================================%
\Oppgave[1] \points*{1}

\begin{figure}[H]
  \centering
  \tikzsetnextfilename{Forkurs-1p-2p-laererutdanning-2016-V-oppgave-1-4}
  \begin{tikzpicture}[scale = 0.5, rotate = 10]
    \edef\circumference{27}
    \pgfmathsetmacro{\X}{(\circumference-3)/4}
    \pgfmathsetmacro{\CD}{(2*\X-5}

    \tkzDefPoint(0,0){A}
    \tkzDefPoint(\X+3,0){B}
    \tkzDefPoint(\X+3,5){C}

    \tkzInterCC[R](A,\X cm)(C,\CD cm) \tkzGetPoints{D}{D2}

    \tkzDrawSegment(A,B)

    \tkzDrawPolygon[thick, maincolorMedium](A,B,C,D)

    \tkzLabelSegment[below](A,B){$x + 3$}
    \tkzLabelSegment[above right](B,C){$5$}
    \tkzLabelSegment[above](C,D){$2x - 5$}
    \tkzLabelSegment[above left](D,A){$x$}
  \end{tikzpicture}
  \caption{}
  \label{fig:del-1-oppgave-4}
\end{figure}

Omkretsen av \cref{fig:del-1-oppgave-4} er $27$. Bestem $x$.


%==============================================================================%
%                                 OPPGAVE 1.5                                  %
%==============================================================================%
\Oppgave[2] \points*{2}

Skriv så enkelt som mulig
%
\begin{equation*}
  4^2 + 4^{-1} \cdot (2^3)^2 + \left( \frac{1}{2} \right)^{-3}
\end{equation*}


%==============================================================================%
%                                 OPPGAVE 1.6                                  %
%==============================================================================%
\Oppgave[2] \points*{2}

\begin{figure}[H]
  \centering
  \begin{subfigure}[b]{0.45\textwidth}
    \centering
    % \tikzsetnextfilename{Forkurs-1p-2p-laererutdanning-2016-V-oppgave-1-6a}
    \begin{tikzpicture}[scale=0.6]
      \tkzInit[xmin=-0.75,xmax=7.9,ymin=-1,ymax=5]
      \tkzClip
      \edef\position{0.5}
      \edef\DE{4} \edef\ST{6} \edef\AB{7} \edef\SR{4.5}
      \pgfmathsetmacro{\trapeziumScale}{\ST/\DE}
      \pgfmathsetmacro{\DC}{\DE*\SR/\ST}
      \pgfmathsetmacro{\xD}{\position*(\AB-\DC)}

      \tkzDefPoint(0,0){A}
      \tkzDefPoint(\AB,0){B}
      \tkzDefPoint(\xD+\DC,\DE){C}
      \tkzDefPoint(\xD,\DE){D}

      \tkzDefLine[orthogonal=through D](A,B) \tkzGetPoint{d}
      \tkzInterLL(A,B)(D,d) \tkzGetPoint{E}

      \tkzMarkRightAngle[thick,color=maincolorMedium,scale=2](D,E,A)
      \tkzDrawSegment[thick,maincolorMedium,dashed](D,E)
      \tkzDrawPolygon[thick,maincolorMedium](A,B,C,D)

      \tkzLabelPoint[below left](A){$A$}
      \tkzLabelPoint[below right](B){$B$}
      \tkzLabelPoint[above right](C){$C$}
      \tkzLabelPoint[above left](D){$D$}
      \tkzLabelPoint[below](E){$E$}
    \end{tikzpicture}
    \caption{}
    \label{fig:del-1-oppgave-6-a}
  \end{subfigure}\hfill%
  \begin{subfigure}[b]{0.55\textwidth}
    \centering
    \tikzsetnextfilename{Forkurs-1p-2p-laererutdanning-2016-V-oppgave-1-6b}
    \begin{tikzpicture}[scale=0.6]
      \tkzInit[xmin=-1.1250,xmax=11.61,ymin=-1.5,ymax=7.5]
      \tkzClip
      \edef\position{0.5}
      \edef\DE{4} \edef\ST{6} \edef\AB{7} \edef\SR{4.5}
      \pgfmathsetmacro{\trapeziumScale}{\ST/\DE}
      \pgfmathsetmacro{\DC}{\DE*\SR/\ST}
      \pgfmathsetmacro{\xD}{\position*(\AB-\DC)}

      \begin{scope}[scale=\trapeziumScale]
        \tkzDefPoint(0,0){A}
        \tkzDefPoint(\AB,0){B}
        \tkzDefPoint(\xD+\DC,\DE){C}
        \tkzDefPoint(\xD,\DE){D}


        \tkzDefLine[orthogonal=through D](A,B) \tkzGetPoint{d}
        \tkzInterLL(A,B)(D,d) \tkzGetPoint{E}

        \tkzMarkRightAngle[thick,color=maincolorMedium,scale=2](D,E,A)
        \tkzDrawSegment[thick,maincolorMedium,dashed](D,E)
        \tkzDrawPolygon[thick,maincolorMedium](A,B,C,D)

        \tkzLabelPoint[below left](A){$P$}
        \tkzLabelPoint[below right](B){$Q$}
        \tkzLabelPoint[above right](C){$R$}
        \tkzLabelPoint[above left](D){$S$}
        \tkzLabelPoint[below](E){$T$}
      \end{scope}
    \end{tikzpicture}
    \vspace*{-0.8cm}
    \caption{}
    \label{fig:del-1-oppgave-6-b}
  \end{subfigure}\hfill%
  \caption{}\label{fig:del-1-oppgave-6}
\end{figure}

De to trapensene ovenfor er formlike. $DE = 4$, $ST = 6$, $AB = 7$ og $SR =
\num{4.5}$.  Bestem arealet av hvert av de to trapesene.


%==============================================================================%
%                                 OPPGAVE 1.7                                  %
%==============================================================================%
\Oppgave[2] \points*{2}

\begin{figure}[H]
  \centering
  \pgfmathsetmacro{\myscale}{\textwidth/18.85cm}
  \tikzsetnextfilename{Forkurs-1p-2p-laererutdanning-2016-V-oppgave-1-7}
  \begin{tikzpicture}[scale=\myscale]
    \edef\AC{12} \edef\BD{10} \edef\CD{6}
    \pgfmathsetmacro{\BC}{sqrt(\BD^2 - \CD^2))}
    \pgfmathsetmacro{\AB}{\AC-\BC}
    \tkzDefPoint(0,0){A}
    \tkzDefPoint(\AB,0){B}
    \tkzDefPoint(\AC,0){C}
    \tkzDefPoint(\AC,\CD){D}

    \tkzMarkRightAngle[color=maincolorMedium, scale=3](B,C,D)
    \tkzDrawSegments[thick, dashed, maincolorMedium](B,C C,D)
    \tkzDrawPolygon[thick, maincolorMedium](A,B,D)

    \tkzLabelPoint[below left](A){$A$}
    \tkzLabelPoint[below](B){$B$}
    \tkzLabelPoint[below right](C){$C$}
    \tkzLabelPoint[above right](D){$D$}
  \end{tikzpicture}
  \caption{}
  \label{fig:del-1-oppgave-7}
\end{figure}


Gitt \cref{fig:del-1-oppgave-7}. $AC = 12$, $BD = 10$, og $CD = 6$. \bigskip

Bestem arealet av $\Delta ABD$.


%==============================================================================%
%                                 OPPGAVE 1.8                                  %
%==============================================================================%
\Oppgave[2]

I en eske ligger det åtte telys. Seks av telysene er røde, og to er hvite.
Tenk deg at du skal ta to telys tilfeldig fra esken.

\begin{oppgaver}
  \Item{1} Bestem sannsynligheten for at du kommer til å ta to røde telys
\end{oppgaver}

\begin{oppgaver}
  \Item{2} Bestem sannsynligheten for at du kommer til å ta ett rødt og ett
    hvitt telys.
\end{oppgaver}


%==============================================================================%
%                                 OPPGAVE 1.9                                  %
%==============================================================================%
\Oppgave[2]

Stian har notert hvor langt han har jogget hver uke de $20$ første ukene i
$2016$. Se \cref{tab:del-1-oppgave-1.9}.

\begin{table}[H]
  \centering
  \begin{tabularx}{\textwidth}{| l | *{10}{Z} |}
    \hline
    \Cellcolor Uke               &  1 & 2 &  3 &  4 & 5 &  6 & 7 &  8 &  9 & 10 \\
    \hline
    \Cellcolor Antall $\si{\km}$ & 11 & 9 & 12 & 22 & 4 & 16 & 8 & 18 & 35 &  3 \\
    \hline
  \end{tabularx} \bigskip

  \begin{tabularx}{\textwidth}{| l | *{10}{Z} |}
    \hline
    \Cellcolor Uke               & 11 & 12 & 13 & 14 & 15 &  16 & 17 &  18 & 19 & 20 \\
    \hline
    \Cellcolor Antall $\si{\km}$ & 30 &  8 &  9 & 39 &  4 &   5 & 25 &   7 & 20 &  2 \\
    \hline
  \end{tabularx}
  \caption{}
  \label{tab:del-1-oppgave-1.9}
\end{table}

\begin{oppgaver}
  \Item{2} Tegn av og fyll ut \cref{tab:del-1-oppgave-1.9a}. Bestem
    gjenomsnittet for det klassedelte datamaterialet.
\end{oppgaver}

\begin{table}[H]
  \centering
  \begin{tabular}{|l | c|}
    \tableHeaders{Lengde ($\si{\km}$)}{Antall uker}
    $\left[\phantom{1}0, \phantom{1}5\right\rangle$ & \\
    $\left[\phantom{1}5,           10\right\rangle$ & \\
    $\left[          10,           20\right\rangle$ & \\
    $\left[          20,           40\right\rangle$ & \\
    \hline
  \end{tabular}
  \caption{}
  \label{tab:del-1-oppgave-1.9a}
\end{table}

\begin{oppgaver}
  \Item{2} Lag et histogram som viser fordeilingen i
    \cref{tab:del-1-oppgave-1.9a}.
\end{oppgaver}

%==============================================================================%
%                                 OPPGAVE 1.10                                 %
%==============================================================================%
\Oppgave[1] \points*{1}

\begin{table}[H]
  \centering
  \begin{tabular}{|S[table-format=4.0] S[table-format=3.1] c|} \hline \Rowcolor
    {År\headerstrut} & {KPI} & {Pris for en kroneis} \\
                1995 &  10.1 & $\phantom{2}1$ krone\phantom{r} \\ \hline
                2015 & 139.8 & $28$ kroner \\
    \hline
  \end{tabular}
  \caption{}
  \label{tab:del-1-oppgave-1.10}
\end{table}

Gitt \cref{tab:del-1-oppgave-1.10}. \bigskip

Vis at prosen for en kroneis har steget mer enn konsumprisinndeksen (KPI) fra
$1955$ til $2015$.


%==============================================================================%
%                                 OPPGAVE 1.11                                 %
%==============================================================================%
\Oppgave[1] \points*{1}

Prisen for en vare er endret tre ganger. Prisen ble først satt ned med
$\SI{10}{\percent}$. Etter en stund ble den igjen satt ned med
$\SI{10}{\percent}$. Senere ble den satt opp med $\SI{20}{\percent}$. \bigskip

Koster varen mer, like mye eller mindre nå enn den gjorde før de tre endringene?
Begrunn svaret ditt.


\ifthenelse{\equal{\isLosningforslag}{False}}{\newpage}{}
%==============================================================================%
%                                 OPPGAVE 1.12                                 %
%==============================================================================%
\Oppgave[3]

I $2016$ er det $\num{4000}$ innbyggere i hver av de to byene $A$ og $B$.
\bigskip

Om by $A$ antar vi følgende:
%
\begin{itemize}
  \item Folketallet vil øke lineært.
  \item I $2020$ vil det være $\num{4080}$ innbyggere i byen.
\end{itemize}

\begin{oppgaver}
  \Item{1} Bestem en modell som viser folketallet $A(x)$ i by $A$, $x$ år
    etter $2016$ ut i fra antakelsene ovenfor.
\end{oppgaver}

Om by $A$ antar vi følgende:
%
\begin{itemize}
    \item Folketallet vil øke eksponentielt i årene som kommer.
    \item I $2017$ vil det være $\num{4080}$ innbyggere i byen.
\end{itemize}

\begin{oppgaver}
    \Item{2} Bestem en modell som viser folketallet $A(x)$ i by $B$, $x$ år
    etter $2016$ ut i fra antakelsene ovenfor.
\end{oppgaver}
